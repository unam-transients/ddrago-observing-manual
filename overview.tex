\chapter{Overview}
\label{chapter:overview}

This chapter aims to provide a one-page overview of DDRAGO to allow you to understand whether it is relevant for your science.

\begin{itemize}
    \item DDRAGO is a two-channel optical imager for the 1.3-meter COLIBRÍ telescope at the Observatorio Astronómico Nacional.
    
    \item Each channel has a $4\mathrm{k}\times4\mathrm{k}$ CCD. The pixel scale is 0.38 \unit{arcsec/pixel} and the field size is 25.9 \unit{arcmin}.

    \item The blue channel has filters that are very close to Pan-STARRS $g$, $r$, and $i$, a wide filter $gri$ that passes from the blue edge of $g$ to the red edge of $i$, and a $B$ filter. 
    
    \item The red channel has filters that are very close to Pan-STARRS $z$ and $y$ and a wide filter $zy$ that passes from the blue edge of $z$ and thus is similar to SDSS $z$.
    
    \item The 10$\sigma$ sensitivity is typically $\AB \approx 20$ in 60 seconds in bright time.

    \item The read time is 2.4 to 7.0 seconds, depending on binning and window.
    
    \item The telescope is located at 115.4646~{\deg} east and 31.0449~{\deg} north, and at an altitude of 2792 meters.

    \item The telescope has an altitude-azimuth mount and can point anywhere above a zenith distance limit of 73.4~{\deg}, which corresponds to an airmass of 3.5. However, the telescope cannot currently track well within 10~{\deg} of the zenith.
    
    \item \cite{Basa-2022} provides more information on the telescope and observatory.

    \item \cite{Langarica-2024} provides more information on the instrument.

\end{itemize}
