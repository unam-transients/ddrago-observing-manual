\chapter{Phase 3}
\label{chapter:phase-3}

This chapter contains information likely to be useful in phase 3 of the observing process: receiving data.

\section{Progress}

The science operations team will inform the technical contact person when their blocks have run.

\section{Data Files}

The data files are created subdirectories whose names are the program, block, and visit identifiers. For example, the files for program 2022A-2001, block 10, visit 0 are created in the directory

\begin{quote}
\verb|2022A-2001/10/0/|
\end{quote}

The base name of each image created by the executor is the UTC time in ISO 8601 basic combined format followed by the channel name (e.g., \verb|C0|, \verb|C1|, \verb|C2|, \ldots), followed by a letter indicating the type of exposure (\verb|o| for object, \verb|f| for flat, \verb|b| for bias, and \verb|d| for dark). 

In COLIBRÍ, \verb|C0| refers to the OGSE test camera, \verb|C1| to the blue channel of DDRAGO, and \verb|C2| to the red channel of DDRAGO.

For example,

\begin{quote}
\verb|20220405T072311C1o|\\
\verb|20220405T072341C1b|\\
\verb|20220405T072351C1f|\\
\verb|20220405T072411C1d|
\end{quote}

For each image there are two files: the full FITS image compressed losslessly with fpack (with suffix \verb|.fz|) and a text version of the header (with suffix \verb|.fits.txt|). In the text version, each record is separated with a newline character.

FITS images compressed with fpack can be read by AstroPy (although you need to select \verb|hdu[1]| rather than \verb|hdu[0]|) and SAOImage DS9.

\section{FITS Header Records}

The FITS header records are largely self-documented by comments. However, these are the most relevant header records for searching for particular data:

\begin{itemize}
\item \verb|DATE-OBS|: The UTC date of the start of the exposure.
\item \verb|INSTRUME|: The channel (e.g., \verb|C0|, \verb|C1|, \verb|C2|, \ldots).
\item \verb|EXPTIME|: The exposure time (seconds).
\item \verb|EXPTYPE|: The exposure type (e.g., \verb|object|, \verb|flat|, \verb|bias|, \verb|dark|).
\item \verb|FILTER|: The selected filter name.
\item \verb|DTDS|: The detector, \verb|SI 1110-167| for the blue CCD and \verb|SI 1110-185| for the red CCD.
\item \verb|BINNING|: The detector binning (pixels).
\item \verb|PRPID|: The proposal identifier (integer).
\item \verb|BLKID|: The block identifier (integer).
\item \verb|VSTID|: The visit identifier (integer).
\item \verb|STRSTRA|: The J2000 RA of the target at the start of the exposure (degrees).
\item \verb|STRSTDE|: The J2000 declination of the target at the start of the exposure (degrees).
\item \verb|STROBHA|: The observed HA of the target at the start of the exposure (degrees).
\item \verb|STROBDE|: The observed declination of the target at the start of the exposure (degrees).
\item \verb|STROBAZ|: The observed azimuth of the target at the start of the exposure (degrees).
\item \verb|STROBZ|: The observed zenith distance of the target at the start of the exposure (degrees).
\item \verb|STROBAM|: The observed airmass of the target at the start of the exposure.
\item \verb|SMNZD|: The observed zenith distance of the Moon at the start of the exposure (degrees).
\item \verb|SMNIL|: The illuminated fraction of the Moon at the start of the exposure.
\item \verb|SMNTD|: The distance between the target and the Moon at the start of the exposure (degrees).
\item \verb|SSNZD|: The observed zenith distance of the Sun at the start of the exposure (degrees).
\end{itemize}

For each record that begins with \verb|S| and refers to the start of the exposure, there is a corresponding record that begins with \verb|E| and refers to the end of the exposure. So, for example, \verb|STROBAM| gives the airmass at the start of the exposure and \verb|ETROBAM| gives the airmass at the end of the exposure.

\section{Downloading Data}

Data are available from transients.astrosen.unam.mx, a server in Ensenada. We shortly hope to have a second server in Mexico City. The technical contact of each program will be provided with an account name and password for the server.

Data are distributed using \verb|rsync|. We recommend reading the man page or a tutorial for \verb|rsync|. Each program is assigned its own “module” in the terminology of \verb|rsync|. 

We have noted deficiencies in the \verb|rsync| command in macOS. We recommend either downloading data from a Linux machine or installing the Homebrew version of rsync (with the command \verb|brew install rsync|, assuming Homebrew is already installed) and then using \verb|/opt/homebrew/bin/rsync| instead of plain \verb|rsync|.

On the rsync server, files are organized in directories as follows:
\begin{quote}\footnotesize
<date>/<program-identifier>/<block-identifier>/<visit-identifier>/
\end{quote}

The date is in the form YYYYMMDD. The program identifiers have the form <semester>-<number>. For early science observations, the block and visit identifiers are determined by the science operations team who typically assign a different block identifier to each target and use visit identified 0 for all science exposures.

We provided to each program raw flats, biases, and darks in addition to their science data. The flats are in program <semester>-0001, biases in <semester>-0002, and darks in <semester>-0003.

Here are some examples of useful commands. We assume the program identifier is 2025A-2081 and the passwords is PASSWORD.

\begin{enumerate}
\item
List all files at the top level in the archive:
\begin{quote}\footnotesize\begin{verbatim}
RSYNC_PASSWORD=PASSWORD \
rsync -a rsync://2025A-2081@ratir.astrosen.unam.mx/2025A-2081/
\end{verbatim}
\end{quote}

\item
List all files at the second level in the archive:
\begin{quote}\footnotesize\begin{verbatim}
RSYNC_PASSWORD=PASSWORD \
rsync -a rsync://2025A-2081@ratir.astrosen.unam.mx/2025A-2081/*/
\end{verbatim}
\end{quote}

\item
List all files in the archive:
\begin{quote}\footnotesize\begin{verbatim}
RSYNC_PASSWORD=PASSWORD \
rsync -a rsync://2025A-2081@ratir.astrosen.unam.mx/2025A-2081/
\end{verbatim}
\end{quote}

\item
Copy all files in the archive into the current directory, maintaining the directory structure:
\begin{quote}\footnotesize\begin{verbatim}
RSYNC_PASSWORD=PASSWORD \
rsync -a rsync://2025A-2081@ratir.astrosen.unam.mx/2025A-2081/ .
\end{verbatim}
\end{quote}

\item
Copy all science files in the archive into the current directory, maintaining the directory structure:
\begin{quote}\footnotesize\begin{verbatim}
RSYNC_PASSWORD=PASSWORD \
rsync -a rsync://2025A-2081@ratir.astrosen.unam.mx/2025A-2081/2025A-2081/ .
\end{verbatim}
\end{quote}

\item
Copy the science files from 2025-02-28 into the current directory, maintaining the directory structure:
\begin{quote}\footnotesize\begin{verbatim}
RSYNC_PASSWORD=PASSWORD \
rsync -a rsync://2025A-2081@ratir.astrosen.unam.mx/2025A-2081/2025A-2081/20250228/ .
\end{verbatim}
\end{quote}

\item
Copy the flat field files from 2025-02-28 into the current directory, maintaining the directory structure:
\begin{quote}\footnotesize\begin{verbatim}
RSYNC_PASSWORD=PASSWORD \
rsync -a rsync://2025A-2081@ratir.astrosen.unam.mx/2025A-2081/2025A-0001/20250228/ .
\end{verbatim}
\end{quote}

\end{enumerate}

\section{Acknowledgements in Publications}

In publications that make use of data acquired with DDRAGO, we request that you cite these papers on the telescope and instrument at an appropriate point:

\begin{itemize}
\item \cite{Basa-2022}
\item \cite{Langarica-2024}

\end{itemize}

We also request that you include this text in the acknowledgments:

\begin{quote}
The data [or some of the data] used in this paper were acquired with the DDRAGO instrument on the COLIBRÍ telescope at the Observatorio Astronómico Nacional on the Sierra de San Pedro Mártir. COLIBRÍ and DDRAGO are funded by the Universidad Nacional Autónoma de México (CIC and DGAPA/PAPIIT IN109418 and IN109224), and CONAHCyT (1046632 and 277901). COLIBRI received financial support from the French government under the France 2030 investment plan, as part of the Initiative d’Excellence d’Aix-Marseille Université-A*MIDEX  (ANR-11-LABX-0060 -- OCEVU and AMX-19-IET-008 -- IPhU), from LabEx FOCUS (ANR-11-LABX-0013), from the CSAA-INSU-CNRS support program, and from the International Research Program ERIDANUS from CNRS. COLIBRÍ and DDRAGO are operated and maintained by the Observatorio Astronómico Nacional and the Instituto de Astronomía of the Universidad Nacional Autónoma de México.
\end{quote}

There is no requirement to include members of the science operations team as authors on any publication that results from these observations in early science time or time awarded by the TACs. However, if you consider that the engineering or science operations teams have been especially helpful, you might consider mentioning them in the acknowledgments.