\chapter{Phase 1}
\label{chapter:phase-1}

This chapter contains information likely to be useful in phase 2 of the observing process: requesting observing time. If you have doubts or special requirements, we recommend that you contact the science operations team.

\section{Mexican Phase 1}

The Mexican astronomical community receives 32.5\% of the time on COLIBRÍ. The process for applying for this time is managed by the Jefatura de Astronomía Observacional of the Instituto de Astronomía and the Comisión de Asignación de Tiempo de Telescopio. For more information, see:

\begin{quote}
\url{https://jao.astroscu.unam.mx/}
\end{quote}

\section{French Phase 1}

The French astronomical community receives 22.5\% of the time on COLIBRÍ. The process for applying for this time is to be determined.

\section{Sensitivity}

A simple exposure-time estimator is available here:

\begin{quote}
\url{https://bit.ly/3HPLyGz}
\end{quote}

The estimator is a spreadsheet in Google Sheets. To use it:
\begin{enumerate}
\item 
Make your own copy by selecting File → Make a Copy in the menus.
\item
In your copy, change the values of the SNR, AB magnitude, and image FWHM to correspond to your science case and expected conditions.
\item 
Note the estimated exposure times in dark and bright conditions.
\end{enumerate}

We do not recommend single exposures of longer than 60 seconds. For more sensitivity, we recommend taking multiple exposures of 60 seconds or less. We offer random dithering in a circle and dithering on a grid with up to 9 points per grid. The default size of the dither pattern is 1 arcmin diameter or 1 arcmin square, but this can be adjusted for each observation.

\section{Overheads}

To account for overheads, conservatively, you should allow:
\begin{itemize}
\item 10 seconds for readout
\item 5 seconds for changing filters
\item 10 seconds for dithering
\item 30 seconds for the initial pointing
\end{itemize}

\section{Calibration}

We routinely take flat field, bias, and dark images. We do not routinely observe photometric standards; we typically calibrate our images from the Pan-STARRS or SkyMapper catalogs.