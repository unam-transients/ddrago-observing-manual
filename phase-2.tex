\chapter{Phase 2}
\label{chapter:phase-2}

This chapter contains information likely to be useful in phase 2 of the observing process: planning observing blocks.

Each program will designate a single technical contact person. All communication with the science operations team will be through this contact person.

Approved observation programs are split into \emph{observing blocks}. We aim to keep individual observing blocks to about 30 minutes of real time (i.e., 24 x 60-second exposures plus overheads). If more data is required, we recommend repeating blocks.

For each observing block, the science operations team will require the following information:

\begin{itemize}
\item Target name. This is strictly optional, but we find it useful.
\item Program number. This will be assigned to each approved program by the science operations team.
\item Target number. We recommend using a distinct target number for each target.
%\item Visit number (default 0). Data files are organized into directories based on the date or observation, program number, block number, and visit number, so the visit number can be used to separate data taken on the same target at different times on the same night.
\item Pointing J2000 coordinates.
\item Filters.
\item Number of exposures in each filter.
\item Exposure time per exposure (default is 60 seconds).
\item If you want to use random dithers, the dither circle diameter (default is 1 arcmin). If you want to dither on a grid, the number of grid points (up to 9) and the size of the grid.
\item Constraints on airmass (by default below airmass 2), hour angle, and date.
\item Whether the block should be run once, multiple times, or repeatedly (i.e., every second night).
\end{itemize}

Once the science operations team has this information, they will program and execute your blocks.

